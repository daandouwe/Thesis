% \bibliography{../src/bibliography}

This chapter introduces a neural Conditional Random Field (CRF) parser that can be used as an alternative proposal distribution for the approximate marginalization of the RNNG. The model is a span-factored CRF that independently predicts scores for labeled spans over the sentence using neural networks. The scores then interact in a chart-based dynamic program, giving a compact description of the probability distribution over parses. This approach combines the efficient exact inference of chart-based parsing, the rich nonlinear features of neural networks, and the global training of a CRF. The chart-based approach allows efficient exact inference alowing for exact decoding and global sampling while the neural features can be complex and can condition on the entire sentence. The CRF training objective [...something about receiving information about all substructures...]. The parser is an adaptation of the chart-based parser introduced in \citet{stern2017minimal} to global CRF training. In \citet{stern2017minimal} the model is trained with a margin-based objective, but given our interest in this model as a proposal distribution we require probabilistic training.

In this chapter:
\begin{itemize}
  \item I present the parser and describe how it is an adaptation of earlier log-linear and neural crf parsing \citep{finkel2008crf,klein2015crf} and the maximum margin trained parser of \citet{stern2017minimal}.
  \item I show how the several inference problems of interest can be solved exactly: prediction, entropy, sampling.
  \item We train the parser in a supervised fashion and show it's adequacy as a parser.
  \item We invesitage the kind of samples that are obtained from this parser, and evaluate how they affect the approximate inference in the RNNG.
\end{itemize}

\section{Model}
  The model is a CRF factored over the labeled spans that form a tree. Let $x$ be a sentence, and $y$ a tree from $\yieldx$. We define a function $\Psi$ that assings nonnegative scores $\Psi(\x, \y)$ and let the probability of a tree be its globally normalized score
  \begin{align}
    \label{eq:crf-model}
    p(\y \mid \x) &= \frac{\Psi(\x, \y)}{Z(\x)},
  \end{align}
  where
  \begin{align*}
    Z(\x) = \sum_{ \y \in \yieldx } \Psi(\x, \y)
  \end{align*}
  is the normalizer, or partition function, that sums over the exponential number of trees availlable for $\x$.

  To allow efficient computation of the normalizer, we let the scoring function $\Psi$ factorize over the parts of $\y$. We consider a tree as a set of labeled spans $\y_a = (A, i, j)$, thus $\y = \{ \y_a \}_{a=1}^A$, where $(A, i, j)$ indicates that $\y$ contains a label $A$ spanning the words $\langle x_{i+1}, \dots, \x_j \rangle$. The value $\Psi(\x, \y)$ is then defined as the product of the nonnegative potentials $\psi(\x, \y_a)$ as
  \begin{align}
    \Psi(\x, \y) = \prod_{a=1}^A \psi(\x, \y_a).
  \end{align}
  The function $\psi$, thus, scores each labeled span $\y_a$ seperately, but conditional on the entire sentence.

  The above model is your typical constituency parsing CRF \citep{finkel2008crf,klein2015crf}, but the factorization over labeled spans, however, was first introduced by \citet{stern2017minimal}. Factorizing a tree over labeled spans makes an even stronger assumption than that typically made of factorizating over anchored rules\footnote{Compare table \ref{tab:spans-rules} for the different conceptions of a tree.}, such as is done in the earlier work. By factorizing over labeled spans, the potential function $\psi$ has no access to information about the direct substructure under the node, such as the child nodes and their split point, and the function can thus rely less on the (local) tree structure and must thus rely more on the surface features of the input sentence. This factorization will, however, greatly reduce the size of the state-space of the dynamic programs, speeding up training and inference, and the burden of the feature function will be carried by a rich neural network parametrization. Together this will make the parser fast yet effective. This will be made clear in section \ref{sect:inference} on inference.

\section{Parametrization}
  The scoring function $\psi$ is implemented with neural networks following \citet{stern2017minimal}. Again, let $\y_a$ denote a labeled span $(A, i, j)$ in a tree $\y$, and let $\psi(\x, \y_a) \geq 0$ be the score of that labeled span for sentence $\x$. These local potentials can only make minimal use of structural information but they can depend on the entire sentence. This suggest the use of bidirectional RNN encodings. Let $\fw_i$ and $\bw_i$ respectively be the vectors computed by a forward and backward RNN for the word in position $i$. The representation of the span $(i, j)$ is the concatenation of the difference between these vector on the endpoints of the span:
  \begin{align}
    \label{eq:span-feature}
    \vecs_{ij} = [\fw_j - \fw_i; \bw_i - \bw_j].
  \end{align}
  The vector $\vecs_{ij}$ represents the words $x_i^j$, and equation \ref{eq:span-feature} is illustrated in figure \ref{fig:span-feature}. The scores the labels of that position are computed from this vector using a feedforward network with output dimension $\reals^{\lvert \Lambda \rvert}$, and the score of label $A$ is given by the index corresponding to it:
  \begin{align}
    \label{eq:potential-function}
    \log \psi(\x, \y_a) = [ \ff( \vecs_{ij} ) ]_{A},
  \end{align}
  where we pretend that $A$ doubles as an index. This architecture is rightly called minimal, but it works surprisingly well: \citet{stern2017minimal} experiment with more elaborate functions based on concatenation of vectors (a strict superset of the minus approach) and biaffine scoring (inspired by \citet{dozat2016deep}), but these function improve marginally, if they do at all.

  \begin{figure}
    \includegraphics[width=\textwidth]{span-encoding.pdf}
    \caption{Representation for the span $(1, 4)$ computed from $\rnn$ encodings. Taken without permission from \citet{stern2018analyis}.}
    \label{fig:span-feature}
  \end{figure}

\section{Inference}
  \label{sect:inference}
  Because the model is span-factored it allows efficient inference. In this section we describe efficient solutions to four related problems:
  \begin{itemize}
    \item Compute the normalizer $Z(\x) = \sum_{ \y \in \yieldx } \prod_{ a=1 }^{ A } \psi( \x, \y_a )$.
    \item Find the best parse $\hat{ \y } = \arg \max_{\y } p(\y  \mid \x)$
    \item Sample a tree $Y \sim P(Y \mid X = x)$. %(\cdot \mid \x)$.
    \item Compute the entropy $H(P(Y \mid X = x))$ over parses for $\x$.
  \end{itemize}
  These problems can be solved by instances of the \textit{inside algorithm} and \textit{outside algorithm} \citep{baker1979trainable} with differentent semirings, an insight we take from semiring parsing \citep{goodman1999semiring}. In the following derivations we will make use of the notion of a \textit{weighted hypergraph} as a compact representation of all parses and their scores \citep{gallo1993directed,klein2004parsing}, and use some of the ideas and notation of \textit{semiring parsing} \citep{goodman1999semiring,eisner2009semirings}. First we describe the structure of the parse forest specified by our CRF parser, and then derrive the particular form of the inside and outside recursions for this hypergraph from the general formulations. We refer the reader to appendix \ref{10-app-crf} for background on these ideas, and the introduction of the notation.

\subsection{Weighted parse forest}
  The hypergraph $G = (V, E)$ specified by the CRF parser has the following structure. The set $V$ consists of the invidual words of a sentence $x$, together with all posible labeled spans over that sentence:
  \begin{align*}
    V = \Big\{ \x_i \; \Big\vert \; 1 \leq i \leq n \Big\} \cup \Big\{ (A, i, j) \; \Big\vert \; A \in \Lambda, 0 \leq i < j \leq n \Big\} \cup \Big\{ (S^{\dagger}, 0, n) \Big\}.
  \end{align*}
  The set of hyperedges $E$ specifies all the ways that adjacent constituents can be combined to form a larger constituent, under a particular grammar. Because we (implicitly) use a normal form grammar that contains \textit{all} possible productions, the set of hyperedges is particularly regular: the set $E$ contains all edges that connect nodes $(B, i, k)$ and $(C, k, j)$ at the tail with $(A, i, j)$ at the head for all $0 \leq i < k < j \leq n$, all edges that connect $\x_i$ to $(A, i, i+1)$, and all nodes can function as top nodes:
  \begin{align*}
    E
      &= \Bigg\{ \Big\langle \Big\{ (B, i, k), (C, k, j) \Big\},  (A, i, j) \Big\rangle \; \Bigg\vert \; A, B, C \in \Lambda, \; 0 \leq i < k < j \leq n \Bigg\}  \\
      &\quad\cup \Bigg\{ \Big\langle \{ \x_i \}, (A, i-1, i) \Big\rangle \; \Bigg\vert \; A \in \Lambda, \; 1 \leq i \leq n \Bigg\}  \\
      &\quad\cup \Bigg\{ \Big\langle \{ (A, 0, n) \}, (S^{\dagger}, 0, n) \Big\rangle \; \Bigg\vert \; A \in \Lambda \Bigg\}
  \end{align*}
  The three kind of edges that make up $E$ are illustrated in figure \ref{fig:crf-edges}. With this structure in place, we are ready to derive the form of the inference algorithm particular to this structure.
  \begin{figure}[h]
    \center
    \begin{tikzpicture}[scale=.6]
      % \documentclass[11pt]{article}
% \usepackage{tikz}
% \usepackage{amsmath,amssymb,amsfonts}
%
% \usetikzlibrary{arrows,calc}
%
% \begin{document}

% \tikzstyle{every node}=[circle, draw, inner sep=0pt, minimum size=7mm]
%
% \tikzstyle{every node}=[circle, draw, inner sep=0pt, minimum size=11mm, node distance =1 cm and 1cm ]

% \begin{tikzpicture}

\node(a){$x_i$} ;
\node(b) at ($(a)+(4.5,0)$){$B_{i}^{k}$} ;
\node(c) at ($(b)+(3,0)$){$C_{k}^{j}$} ;
\node(d) at ($(c)+(4.5,0)$){$A_{0}^{n}$} ;

\node(A1) at ($(a)+(0,3)$){$A_{i-1}^{i}$} ;
\node(A2) at ($(b)+(1.5,3)$){$A_{i}^{j}$} ;
\node(S) at ($(d)+(0,3)$){$S^{\dagger}_{0}^{n}$} ;

\draw[->] (a) to [in=-90,out=90](A1);

\draw[->] (b) to [in=-90,out=90](A2);
\draw[->] (c) to [in=-90,out=90](A2);

\draw[->] (d) to [in=-90,out=90](S);

% \end{tikzpicture}
%
% \end{document}

    \end{tikzpicture}
    \caption{The three types of edges making up the hypergraph of example \ref{ex:crf}.}
    \label{fig:crf-edges}
  \end{figure}
  We connect a semiring $\mathcal{K}$ to the hypergraph by defining the the weight function as $\omega: E \to \mathbb{K}$, and by accumulating the weights with its binary operations. The function $\omega$ that asigns weights to the edges will be given either by the function $\psi$ or $\log \circ \psi$, depending on the semiring used. Because of the factorization that we assumed, the function $\omega$ has a very particular property: the function only depends on the \textit{head} of the edge. Let $e$ be an edge of the form $\langle \{ (B, i, k), (C, k, j) \},  (A, i, j) \rangle$, for example, then
  \begin{align}
    \omega(e) = \omega(\langle \{ (B, i, k), (C, k, j) \},  (A, i, j) \rangle) = \omega((A, i, j)),
  \end{align}
  which we write as $\omega(A,i,j)$ for simplicity. This fact will allow us to greatly rewrite the recursion that follow.

\subsection{Inside recursion}
  The inside recursion computes quantities $\alpha(A,i,j)$ for all labels $A \in \Lambda$ and all spans $0 \leq i < j \leq n$. The quantity computed depends on the semiring used. In this section we derive the inside recursion specific to our hypergraph from the general result given. What we will show in particular is that the factorization over labeled spans allows the formula to simplify greatly, speeding up the computation significantly.

  Let $\mathcal{K}$ be some semiring with binary operations $\oplus$ and $\otimes$ and identity elements $\hat{0}$ and $\hat{1}$. The inside recursion is given by the formula \citep{goodman1999semiring}
  \begin{align*}
    \alpha(v) =
      \begin{cases}
        \hat{1}  &  \mbox{if } I(v) = \varnothing  \\
        \displaystyle\bigoplus_{e \in I(v)} \omega(e) \otimes \displaystyle\bigotimes_{u \in T(e)} \alpha(u)  & \mbox{otherwise.}
      \end{cases}
  \end{align*}

  At a node $v = (A, i, i+1)$ that spans one word $x_i$, the inside value is just the weight of the single edge incoming from that word:
  \begin{align}
      \label{eq:inside-base}
      \alpha(A, i, i+1) = \omega(A, i, i+1) \otimes \alpha(x_i) = \omega(A, i, i+1),
  \end{align}
  for $A \in \Lambda$, for all $0 \leq i < n$. We used the fact that $\alpha(x_i) = \hat{1}$, which follows from the fact that there are no arrows incoming at $\x_i$.

  For a general node $\alpha(A, i, j)$, $j > i + 1$, we observe that all the incoming edges have at the tail the nodes $(B, i, j)$ and $(C, k, j)$, for all $B, C \in \Lambda$ and $i < k < j$. The sum over edges thus reduces to independent sums over $B$, $C$, and $k$, and the product over the inside values at the tail reduces to the product of values $\alpha(B, i, k)$ and $\alpha(C, k, j)$. The form of $\omega$ allows us to rewrite this greatly as
  \begin{align}
    \label{eq:inside}
    \alpha(A, i, j)
      &= \bigoplus_{B \in \Lambda} \bigoplus_{C \in \Lambda} \bigoplus_{k=i+1}^{j-1} \omega(A, i, j) \otimes \alpha(B,i,k) \otimes \alpha(C,k,j) \nonumber \\
      &= \omega(A, i, j) \otimes \bigoplus_{k=i+1}^{j-1} \bigoplus_{B \in \Lambda} \alpha(B,i,k) \otimes \bigoplus_{C \in \Lambda} \alpha(C,k,j) \nonumber \\
      &= \omega(A, i, j) \otimes \bigoplus_{k=i+1}^{j-1} \sigma(i,k) \otimes \sigma(k,j),
  \end{align}
  where we've introduced the notational abbreviation
  \begin{align*}
      \sigma(i,j) &= \bigoplus_{A \in \Lambda} \alpha(A,i,j).
  \end{align*}
  Looking at \ref{eq:inside} we can see the marginalized values $\sigma(i, j)$ are all that are needed for the recursion. This suggest simplifying the recursion even further as
  \begin{align}
    \label{eq:inside-simplified}
    \sigma(i, j)
      &= \bigoplus_{A \in \Lambda} \alpha(A,i,j) \nonumber \\
      &= \Bigg[ \bigoplus_{A \in \Lambda} \omega(A, i, j) \Bigg] \otimes \Bigg[\bigoplus_{k=i+1}^{j-1} \sigma(i,k) \otimes  \sigma(k,j) \Bigg],
  \end{align}
  where we put explicit brackets to emphasize that independence of the subproblems of labeling and splitting.

\subsection{Outside recursion}
  The outside algorithm computes the quantities $\beta(A,i,j)$ for all labels $A \in \Lambda$ and all spans $0 \leq i < j \leq n$. For the labels spanning the entire sentence, the outside value is defined as
  \begin{align*}
    \beta(A, 0, n) =
    \begin{cases}
      \hat{1}, & \text{ if } A = \text{S}^{\dagger}  \\
      \hat{0}, & \text{otherwise}.
    \end{cases}
  \end{align*}
  To compute $\beta(A, i, j)$ in the general case we need to sum over all possible upward binary expansions that $A$ can be a part of, recursively making use of the outside values already computed. This means we sum over expansions $(B \to C \; A, k, i, j)$ for all labels $B$ and $C$ and left endpoints $1 \leq k < i$, and sum over expansions $(B \to A \; C, i, j, k)$ for all labels $B$ and $C$ and right endpoints $j < k \leq n$. This corresponds to the following recursive expression that uses the outside values of the superparts:
  \begin{align*}
    \beta(A, i, j)
      &= \bigoplus_{B \in \Lambda} \bigoplus_{C \in \Lambda} \bigoplus_{k=1}^{i-1} \psi(B, k, j) \otimes \alpha(C, k, i-1) \otimes \beta(B, k, j) \\
        &\qquad \oplus \bigoplus_{B \in \Lambda} \bigoplus_{C \in \Lambda} \bigoplus_{k=j+1}^{n} \psi(B, i, k) \otimes \beta(B, i, k) \otimes \alpha(C, j+1, k) \\
      &=  \bigoplus_{k=1}^{i-1}  \Bigg[ \bigoplus_{B \in \Lambda} \psi(B, k, j)  \otimes \beta(B, k, j) \Bigg] \otimes \Bigg[ \bigoplus_{C \in \Lambda} \alpha(C, k, i-1) \Bigg] \\
        &\qquad \oplus \bigoplus_{k=j+1}^{n}  \Bigg[ \bigoplus_{B \in \Lambda}  \psi(B, i, k) \otimes \beta(B, i, k) \Bigg] \otimes  \Bigg[  \bigoplus_{C \in \Lambda} \alpha(C, j+1, k) \Bigg] \\
      &=  \bigoplus_{k=1}^{i-1}  \sigma'(k, j) \otimes \sigma(k, i-1) \oplus \bigoplus_{k=j+1}^{n} \sigma'(i, k) \otimes  \sigma(j+1, k) \\
  \end{align*}
  where
  \begin{align*}
      \sigma(i, j) &= \bigoplus_{A \in \Lambda} \alpha(A, i, j) \\
      \sigma'(i, j) &= \bigoplus_{A \in \Lambda} \psi(A, i, j) \beta(A, i, j)
  \end{align*}

\subsection{Solutions}
  Equiped with the above two recursions and handful of semirings we can provide the solutions promised at the outset of this section.

  \paragraph{Normalizer}
    When we intantiate the inside recursion with the real semiring, the value of $\alpha$ at the root is the normalizer:
    \begin{align*}
      \alpha(\text{S}^{\dagger}, 0, n) = Z(\x),
    \end{align*}
    and when we instantiate the inside recursion with the log-real semiring we obtain the log-normalizer
    \begin{align*}
      \alpha(\text{S}^{\dagger}, 0, n) = \log Z(\x).
    \end{align*}

  \paragraph{Parse}
    To find the viterbi tree $\hat{ \y } = \arg \max_{\y } p(\y  \mid \x)$ and its probability $p(\hat{\y} \mid \x)$ we use the Viterbi semirings ($\cf$ examples \ref{ex:vit-weight} and \ref{ex:vit-derivation} in appendix \ref{A3-crf}). We take equation \ref{eq:inside-simplified} and use the Viterbi semiring operations to derive that the value of the best subtree spanning words $i$ to $j$ is given by
    \begin{align}
      \label{eq:viterbi-score}
      \sigma(i,j)
        &= \max_{A} [ \log \psi(A, i, j) ] + \max_{k} [\sigma(i,k) + \sigma(k,j)].
    \end{align}
    The value $\log Psi(\x, \hat{\y})$ is then given by $\sigma(0, n)$, and can be normalized with $\log Z(x)$ to give the probability. The best label and splitpoint $\hat{A}$ and $\hat{k}$ for the span $(i, j)$ are obtained by using the argmax:
    \begin{align}
      \label{eq:viterbi-tree}
      \hat{A} &= \argmax_{ A  } \log \psi(A, i, j)  \\
      \hat{k} &= \argmax_{ k } \sigma(i, k) + \sigma(k, j),
    \end{align}
    and the best tree $\hat{y}$ is found by following back from the root down to the leaves the best splits and labels.

  \paragraph{Sample}
    Samples can be obtained by recursively sampling edges, starting at the root node $(S^{\dagger}, 0, n)$. The probability of an edges is proportional to the weight under that edge: this is precicely the inside value $\alpha$ computed in the real-semiring. An edge $e = \langle \{ u, w \}, v \rangle$, with $u = (B, i, k)$, $w = (C, k, j)$ and $v = (A, i, j)$, has probability
    \begin{align}
      \label{eq:sample}
      P(e)
        &= \frac{\omega(e) \otimes \bigotimes_{u \in T(e)} \alpha(u)}{\alpha(v)}  \nonumber  \\
        &= \frac{\psi(A, i, j) \, \alpha(B, i, k) \, \alpha(C, k, j)}{\alpha(A, i, j)}  \nonumber  \\
        &\propto \alpha(B, i, k) \, \alpha(C, k, j),
    \end{align}
    where the edge-weight $\psi(A, i, j)$ is absorbed into the normalizing constant because it is the same for all edges.

  \paragraph{Entropy}
    To compute the entropy $H(P(Y \mid X = x))$ we need to first introduce the notion of the \textit{maginal probablity} of a node. The marginal of a node $v = (A, i, j)$ in a hypergraph is the probability that it occurs in a tree $\y$ for the sentence $\x$ as governed by the distribution $p$ that we defined on it. Let $V$ be a random variable with the hypergraph nodes $\mathcal{V}$ as sample space,\footnote{Please excuse us for overloading notation here...} and define
    \begin{align}
      P(V = v \mid X = \x)
        &\defeq \expect_Y[ \indicator_{ \{ v \in Y \} } ]  \nonumber \\
        &= \sum_{ \y \in \yieldx } p(\y \mid \x) \indicator_{ \{ v \in y \} }.
    \end{align}
    The marginals can be computed from the inside and outside values computed in the real semiring as
    \begin{align}
      P(V = v \mid X = \x) = \frac{\alpha(A, i, j) \, \beta(A, i, j)}{Z(\x)},
    \end{align}
    a result from \citep{goodman1999semiring}.\footnote{This can also be seen by noting that the product of $\alpha(A, i, j)$ and $\beta(A, i, j)$ is the sum over all trees that contain the node $v$, and $Z(\x)$ the sum over all trees in general.} The entropy can be written as with these marginals as
    \begin{align}
      H(P(Y \mid X = x))
        &= - \sum_{ \y \in \yieldx } p(\y \mid \x) \log p(\y \mid \x)  \nonumber \\
        &= \log Z(\x) - \sum_{ \y \in \yieldx } p(\y \mid \x) \sum_{v \in \y} \log \psi(\x, v)  \nonumber \\
        &= \log Z(\x) - \sum_{ \y \in \yieldx } p(\y \mid \x) \sum_{ v \in V } \indicator_{ \{ v \in y \} } \log \psi(\x, v)  \nonumber \\
        &= \log Z(\x) - \sum_{ v \in \mathcal{V} } \log \psi(\x, v)  \sum_{ \y \in \yieldx } \indicator_{ \{ v \in y \} } p(\y \mid \x)  \nonumber \\
        &= \log Z(\x) - \sum_{ v \in \mathcal{V} } \log \psi(\x, v)  P(V = v \mid X = \x)  \nonumber \\
        &=  \log Z(\x) - \expect_V [ \log \psi(\x, V) ].
    \end{align}

\section{Experiments}
  We perform three types of experiments with the CRF parser:
  \begin{itemize}
    \item We show that the model is a good supervised parser. We train the model supervised on the PTB and show the f-score on the PTB test set.
    \item We evaluate the joint RNNG with samples from the CRF parser. We compare the perplexity and fscore with RNNG case.
    \item We evauluate `how good the model is' as a sampler.
  \end{itemize}

\subsection{Supervised model}
  We investigate the following.
  \begin{itemize}
    \item We have some optimization and hyperparameter choices here. The original paper uses Adam with 0.001 and a LSTM of dimension 250, which gives the model around 2.5 million parameters. For the discriminative RRNG we use SGD with 0.1, and hidden sizes of 128 gives the model around 800,000 parameters.
    \item I suggest two experiments: (1) use the default setting from \citep{stern2017minimal} and (2) use the settings for the RNNG with a hidden size to match the 800,000 parameters.
  \end{itemize}

\subsection{Proposal model}
  We investigate the following:
  \begin{itemize}
    \item We evaluate validation F-score and perplexity.
    \item We evaluate F-score with 100 samples (as many proposal trees as possible).
    \item We evaluate perplexity with varying number of samples: 1 (argmax), 10, 20, 50, 100 (default). The peplexity evaluation with the argmax prediction gives an impression of the uncertaty in the model \citep{buys2018exact}.
    \item  We perform learning rate decay and model selection based on a development score computed with the samples from the discriminative RNNG. Undecided: should we train a separate joint RNNG with CRF samples?
  \end{itemize}

\subsection{Sampler}
  We investigate the following:
  \begin{itemize}
    \item We asses the conditional entropy of the model. This is most quantitative. Recall that conditional entropy is defined as
    \begin{equation}
      \text{H}(Y|X) = \sum_{x \in \mathcal{X}} p_X(x)\text{H}(Y|X = x),
    \end{equation}
    where
    \begin{equation}
      \text{H}(Y|X = x) = - \sum_{y \in \mathcal{Y}} p_{Y|X}(y|x) \log p_{Y|X}(y|x).
    \end{equation}
    The quantity $\text{H}(Y|X = x)$ can computed exactly with the CRF parser. We estimate the quantity $\text{H}(Y|X)$ by a sum over the development dataset. For the probabilities $p_X(x)$ we use the marginalized probabilities of the joint RNNG (with samples from the CRF parser $p_{Y|X}$).
    \item We asses for some cherry picked sentences. This is more qualitative. These sentences should be difficult or ambiguous. Or they can be ungramatical when taken from the syneval dataset. We can evaluate their entropy, and the diversity of samples, for example to see if there are clear modes. We can make violinplots of the probabilities of the samples. We can compute the f-scores of the samples compared with the argmax tree.
  \end{itemize}

\section{Related work}
  Here I describe related work, and in particular earlier approaches to (neural) CRF-parsing.
  \begin{enumerate}
    \item Of course \citep{stern2017minimal}
    \item CRFs \citep{sutton2012crf}
    \item CRF parsing with linear and nonlinear features \citep{finkel2008crf,klein2015crf}
    \item Attempts to simplify the grammar and thus the state-space of the dynamic program \citep{hall2014less}.
    \item Recent extension of \citet{stern2017minimal}, with same model but different features \citep{kitaev2018attentive}.
  \end{enumerate}

\subsection{Motivation}
  The model regards a constituency tree as a collection of \textit{labeled spans} over a sentence. Earlier CRF models for constituency parsing, both log-linear and neural, factorize trees over \textit{anchored rules} \citep{finkel2008crf,klein2015crf}. This puts most of the expressiveness of the model in the state space of the dynamic program, modelling interactions between subparts of the trees through their interaction in the rules, instead of at the feature level. The model in \citet{stern2017minimal} removes part of this structure, and puts more expressiveness in the input space by using rich neural feature representations conditioning on the entire sentence. The discrete interaction between the local scores remains only at level of labeled spans. This dramatically improves the speed of this model, which will become evident in the next section.

  Chart based parsing has moved from grammar to features: the work of \citet{hall2014less} shows how the log-linear CRF model of \cite{finkel2008crf} can work with bare unnanotated grammars, when relying more on the surface features of the sentence, and \citet{klein2015crf} show how these surface features can be replaced with dense neural features. The work of \citep{stern2017minimal} moves one step further: the grammar is dispensed with altogether, making the model span-factored, and the scoring function is given the full power of neural networks.

  Contrast this with generative parsing based on treebank grammars, where features are not available because the models are not conditional. Instead, these models rely entirely on detailed rule information: basic treebank grammars do not parse well because the rules provide too little context, and good results can only be obtained by enriching grammars. The independence assumptions in the grammar are thus typically weakened, not strengtehened. Such approaches lexicalize rules \citep{collins2003head}, annotate the rule with parent and sibling labels \citep{klein2003accurate}, or automatically learn refinements of nonterminal categories \citep{petrov2006learning}.

  The closest predecessor to our model is the neural CRF parser of \citet{klein2015crf}, which predicts local potential for anchored rules using a feedfoward network. This differs from our approach in two ways. Their method requires a grammar, extracted from a treebank beforehand, whereas our approach implictly assumes all rules are possible rules in the grammar. Secondly, their scoring function conditions only on the parts of the sentence directly under the rule, dictated by the use of a feedforward network, whereas our scoring function computes a score bassed on representations computed from the entire sequence.

  Earlier work on CRF parsing consider a tree as a collection of anchored rule productions productions \cite{finkel2008crf,klein2015crf}, and hence define the score of a tree as the product over clique potentials on anchored rules:
  \begin{align}
    \log\psi(A \to B \;C, i, k, j) = \log\psi(A, i, j)\\
  \end{align}
  discarding the rest of the span information. The function is then defined as
  \begin{align}
    \label{eq:span-score}
    \log\psi(A, i, j) &\defeq s(i, j, A),
  \end{align}
  Note that the potential function as defined in \ref{eq:rule-score} disregards most of the information in a binary rule. In particular we see that $B$, $C$ and $k$, the labels and split-point of the children, are discarded.
