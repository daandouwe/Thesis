% \bibliography{../src/bibliography}

The derivation of the inference algorithms in chapter \ref{04-crf} makes use of the notion of a \textit{weighted hypergraph} as a compact representation of all parses and their scores \citep{gallo1993directed,klein2004parsing}, and also makes use some of the ideas and notation of semiring parsing \citep{goodman1999semiring,eisner2009semirings}.


\section{Hypergraph}
  We use a \textit{backward hypegraph} to compactly represent all possible trees over a sentence under a grammar, an idea introduced in \citet{klein2004parsing}. We call such hypergraph a parse forest.

  \begin{definition}{}
    A \textit{directed hypergraph} is a pair $G = (V, E)$, consisting of a set of nodes $V$, and a set of directed hyperedges $E \subseteq 2^V \times 2^V$ that connect a set of nodes at the \textit{tail} of the arrow to a set of node at the \textit{head} of the arrow.
  \end{definition}

  \begin{definition}{}
    A \textit{backward-hypergraph} is a directed hypergraph where the hyperedges $E \subseteq 2^V \times V$ connect to a single node. For a node $v \in V$ the set $I(v) \subseteq E$ denotes the set of edges incoming at $v$ and the set $O(v) \subseteq E$ denotes the set of edges outgoing from $v$, $\ie$ the edges for which $v$ is in the tail. For an edge $e$ we write $T(e) \subseteq V$ for the set of nodes in the tail of $e$, and define $H(e) \in V$ as the node at the head of the edge.
  \end{definition}

  \begin{definition}{}
     A \textit{weighted hypergraph} is any hypergraph $G$ equided with a weight function $\omega : E \to K$ that to each edge assigns a weight from a weight-set $K$. Typically $K$ the set of real numbers, or the set of integers.
  \end{definition}

  \begin{example}{}
    The hypergraph specified by the CRF parser in chapter \ref{04-crf} has the following structure. Let the set $V$ consists of the invidual words of a sentence $x$, together with the all posible labeled spans over that sentence:
    \begin{align*}
      V = \Big\{ \x_i \; \Bigg\vert \; 1 \leq i \leq n \Big\} \cup \Big\{ (A, i, j) \; \Bigg\vert \; A \in \Lambda, 0 \leq i < j \leq n \Big\} \cup \Big\{ (S^{\dagger}, 0, n) \Big\}.
    \end{align*}
    The edge set $E$ will specify all the ways that adjacent constituents can be combined to form a larger constituent, under a particular grammar. Because we (implicitly) use a normal form grammar that contains \textit{all} possible productions, the set of hyperedges is particularly regular: the set $E$ contains all edges that connect nodes $(B, i, k)$ and $(C, k, j)$ at the tail with $(A, i, j)$ at the head for all $0 \leq i < k < j \leq n$, all edges that connect $w_i$ to $(A, i, i+1)$, and all nodes can function as top nodes:
    \begin{align*}
      E
        &= \Bigg\{ \Big\langle \Big\{ (B, i, k), (C, k, j) \Big\},  (A, i, j) \Big\rangle \; \Bigg\vert \; A, B, C \in \Lambda, \; 0 \leq i < k < j \leq n \Bigg\}  \\
        &\quad\cup \Bigg\{ \Big\langle \{ w_i \}, (A, i-1, i) \Big\rangle \; \Bigg\vert \; A \in \Lambda, \; 1 \leq i \leq n \Bigg\}  \\
        &\quad\cup \Bigg\{ \Big\langle \{ (A, 0, n) \}, (S^{\dagger}, 0, n) \Big\rangle \; \Bigg\vert \; A \in \Lambda \Bigg\}
    \end{align*}
    Figure \ref{fig:hypergraph} shows a fragment of this hypergraph for an example sentence.

  \end{example}

  The edges that direct to the special node $(S^{\dagger}, 0, n)$ make the hypergraph \textit{rooted}: each \textit{hyperpath} with the set of all words at the source ends at this one node. Formally defining a hyperpath from a set of nodes $W \subseteq V$ to a node $v \in V$ is a little cumbersome, but informally, a hyperpath is a set of edges otained by the following recursive procedure: starting at $v$, follow a single incoming hyperarc $e$ backwards, and repeat this for all nodes in $T(e)$; stop when all nodes in $W$ have been passed.

\begin{figure}[h]
  \center
  \begin{tikzpicture}[scale=.6]
    % \documentclass[11pt]{article}
% \usepackage[upright]{fourier}
% \usepackage{tikz}
% \usepackage{amsmath,amssymb,amsfonts}
%
% \usetikzlibrary{arrows,calc}
%
% % \newcommand{\boundellipse}[3]% center, x rad, y rad
% % {(#1) ellipse [x radius=#2,y radius=#3]
% % }
%
% \begin{document}
%
% % \tikzstyle{every node}=[circle, draw, inner sep=0pt, minimum size=13mm,
% % node distance =1 cm and 1cm ]
%
% \begin{tikzpicture}
\node(a){\textit{The}} ;
\node(b) at ($(a)+(2,0)$){\textit{very}} ;
\node(c) at ($(b)+(2,0)$){\textit{hungry}} ;
\node(d) at ($(c)+(2,0)$){\textit{cat}} ;
\node(e) at ($(d)+(2,0)$){\textit{meows}} ;
\node(f) at ($(e)+(2,0)$){\textit{.}} ;

\node(A) at ($(a)+(0,1.5)$){$\varnothing_{0,1}$} ;
\node(B) at ($(A)+(2,0)$){$\varnothing_{1,2}$} ;
\node(C) at ($(B)+(2,0)$){$\varnothing_{2,3}$} ;
\node(D) at ($(C)+(2,0)$){$\varnothing_{3,4}$} ;
\node(E) at ($(D)+(2,0)$){$VP_{4,5}$} ;
\node(F) at ($(E)+(2,0)$){$\varnothing_{5,6}$} ;

\node(ADJP) at ($(B)+(0.8,2)$){$ADJP_{1,3}$} ;

\node(EMPTY) at ($(ADJP)+(0.5,2)$){$\varnothing_{1,4}$} ;
\node(NP) at ($(ADJP)+(2,2)$){$NP_{1,4}$} ;

\node(NP2) at ($(A)+(2,7)$){$NP_{0,4}$} ;

\node(EMPTY2) at ($(c)+(0,11)$){$\varnothing_{0,5}$} ;
\node(S) at ($(d)+(0,13)$){$S_{0,6}$} ;


\draw[->] (a) to [in=-90,out=90](A);
\draw[->] (b) to [in=-90,out=90](B);
\draw[->] (c) to [in=-90,out=90](C);
\draw[->] (d) to [in=-90,out=90](D);
\draw[->] (e) to [in=-90,out=90](E);
\draw[->] (f) to [in=-90,out=90](F);

\draw[->] (B) to [in=-90,out=90](ADJP);
\draw[->] (C) to [in=-90,out=90](ADJP);

\draw[->] (ADJP) to [in=-90,out=90](EMPTY);
\draw[->] (D) to [in=-90,out=90](EMPTY);

\draw[dashed,->] (ADJP) to [in=-90,out=90](NP);
\draw[dashed,->] (D) to [in=-90,out=90](NP);

\draw[->] (A) to ++(-0.2,3) to [in=-90,out=90](NP2) ;
\draw[->] (EMPTY) to [in=-90,out=90](NP2) ;

\draw[dashed,->] (A) to [in=-90,out=90](NP2) ;
\draw[dashed,->] (NP) to [in=-90,out=90](NP2) ;

\draw[->] (NP2) to [in=-90,out=90](EMPTY2);
\draw[->] (E) to [in=-90,out=90](EMPTY2);

\draw[->] (EMPTY2) to [in=-90,out=90](S);
\draw[->] (F) to [in=-90,out=90](S);

% \end{tikzpicture}
%
% \end{document}

  \end{tikzpicture}
  \caption{A fraction of a parse hypergraph showing two possible parses.}
  \label{fig:hypergraph}
\end{figure}


\section{Semiring}
  We use \textit{semirings} to compute various quantities of interest over a weighted hypergraph.

  \begin{definition}{}
    A \textit{semiring} is an algebraic structure
    \begin{align*}
      \mathcal{K} = \langle \mathbb{K}, \oplus, \otimes, \bar{0}, \bar{1} \rangle,
    \end{align*}
    over a field $\mathbb{K}$, with additive and multiplicative operations $\oplus$ and $\otimes$, and additive and multiplicative identities $\bar{0}$ and $\bar{1}$. A semiring is equivalent to a ring\footnote{Perhaps the most common algebraic structure around: the real numbers with regular addition and multiplication form a ring.}, without the requirement of an additive inverse for each element.
  \end{definition}

  Some of the semirings relevant to our discussion are the following:
  \begin{example}{}
    The \textit{real semiring},
    \begin{align*}
      \langle \reals_{\geq 0}, +, \times, 0, 1 \rangle,
    \end{align*}
    defined over the nonnegative reals, with regular addition and multiplication is a semiring. Note that the additive inverse is missing for all elements ($\ie$ the negative reals).
  \end{example}

  \begin{example}{}
    The \textit{boolean semiring},
    \begin{align*}
      \langle \{ \top, \bot \}, \vee, \wedge, \top, \bot  \rangle
    \end{align*}
    defined over truth values $\top$ (True, or 1), and $\bot$ (False, or 0).
  \end{example}

  \begin{example}{}
    The \textit{log-real semiring},
    \begin{align*}
      \langle \reals \cup \{ \infty \}, \oplus, +, -\infty, 0 \rangle,
    \end{align*}
    defined over the reals extended including $\infty$, with addition defined as the logarithmic sum\footnote{Also known as \textit{log-sum-exp}, or \textit{log-add-exp}.}
    \begin{align*}
      a \oplus b = \log( e^{a} + e^{b} ),
    \end{align*}
   and multiplication is defined as regular addition.
  \end{example}

  \begin{example}{}
    The \textit{max-tropical semiring}, or \textit{Viterbi} semiring\footnote{This naming will become clear.}
    \begin{align*}
      \langle \reals_{\geq 0} \cup \{ \infty \}, \max, +, -\infty, 0 \rangle,
    \end{align*}
    is the log-real semiring that uses the $\max$ operator for addition.
  \end{example}

\subsection{Semiring parsing}
  A semiring $\mathcal{K}$ can be connected to a weighted hypergraph by defining the function $\omega$ over its field $\mathbb{K}$, and by accumulating the weights with its binary operations. When the hypergraph represents a parse forest, we are in the realm of \textit{semiring parsing} \citep{goodman1999semiring}.

  The key observation is that these quantities of can be computed by the one recursion, instantiated by a different semiring.

  \begin{definition}{} Let $G = (V, E, \omega)$ be a weighted hypergraph, with $\omega$ defined over a semiring $\mathcal{K}$. We define the weight of the derrivation $D \subseteq E$ as the product of the weights of the edges:
  \begin{align*}
    \bigotimes_{e \in D} \omega(e).
  \end{align*}
  Let $\mathcal{D} \subseteq 2^E$ be the set of all derivations in the hypergraph $G$. Then the total weight of the hypergraph under $\omega$ is defined as the sum of the weights of all the derrivations in it
  \begin{align*}
    \bigoplus_{D \in \mathcal{D}} \bigotimes_{e \in D} \omega(e).
  \end{align*}
  \end{definition}

  The function $\omega$ assigns a weights to the derrivations encoded by factoring over the edges that make up those derivations. In the CRF parser, the weight function $\omega$ corresponds to the local potential function $\Psi$; the weight of a derivation is the product of those weights; and the set $\mathcal{D}$ corresponds to the set of admissible trees $\yieldx$ for a sentence $\x$.

  The following definitions follow \citet{eisner2009semirings}.
  \begin{definition}{} The \textit{inside value} $\alpha(v)$ at a node $v \in V$ accumulates the weight of all the paths that converge at that node. The accumulation is relative to a semiring, and is defined as
  \begin{align*}
    \alpha(v) =
      \begin{cases}
        \hat{1}  &  \mbox{if } I(v) = \varnothing  \\
        \displaystyle\bigoplus_{e \in I(v)} \omega(e) \otimes \displaystyle\bigotimes_{u \in T(e)} \alpha(u)  & \mbox{otherwise.}
      \end{cases}
  \end{align*}
  The value $\alpha(v)$ at the root node $v$ is the sum of the weight of all the derivations in the hypergraph. The recursion can solved by visiting the nodes in $V$ in topological order.
  \end{definition}

  \begin{definition}{} The \textit{outside value} $\beta(v)$ at a node $v \in V$ accumulates over the weights of all the paths that head out from $v$. The accumulation is relative to a semiring, and is defined as:
  \begin{align*}
    \beta(v) =
      \begin{cases}
        \hat{1}  & \mbox{if } O(v) = \varnothing \\
        \displaystyle\bigoplus_{e \in O(v)} \omega(w) \otimes \beta(H(e)) \otimes \displaystyle\bigotimes_{ \substack{ w \in T(e) \\ w \neq u } } \alpha(w)  & \mbox{otherwise.}
      \end{cases}
  \end{align*}
  The recursion can solved by visiting the nodes in $V$ in reverse topological order.
  \end{definition}

  When the parse forest is highly regular these equations simplify somewhat:
  \begin{align}
    \alpha(v) =
      \begin{cases}
        \hat{1}  &  \mbox{if } I(v) = \varnothing  \\
        \displaystyle\bigoplus_{e \in I(v)} \omega(e) \otimes \displaystyle\bigotimes_{u \in T(e)} \alpha(u)  & \mbox{otherwise.}
      \end{cases}
  \end{align}
  The value $\alpha(v)$ at the root node $v$ is the sum of the weight of all the derivations in the hypergraph. The recursion can solved by visiting the nodes in $V$ in topological order.
  \end{definition}

  \begin{definition}{} The \textit{outside value} $\beta(v)$ at a node $v \in V$ accumulates over the weights of all the paths that head out from $v$. The accumulation is relative to a semiring, and is defined as:
  \begin{align*}
    \beta(v) =
      \begin{cases}
        \hat{1}  & \mbox{if } O(v) = \varnothing \\
        \displaystyle\bigoplus_{e \in O(v)} \omega(w) \otimes \beta(H(e)) \otimes \displaystyle\bigotimes_{ \substack{ w \in T(e) \\ w \neq u } } \alpha(w)  & \mbox{otherwise.}
      \end{cases}
  \end{align*}
  The recursion can solved by visiting the nodes in $V$ in reverse topological order.
  \end{definition}


By instanstiating these algorithms with different semirings we solve a whole number of problems:
\begin{itemize}

    \item When we intantiate the inside recursion and the outside recursion with the \textit{real semiring}, the algorithms recude to the classical inside-outside algorithm \citep{baker1979trainable}, and the values $\alpha(v)$ and $\beta(v)$ are the inside and outside values. In particular, the value at the root is the normalizer $Z(\x)$. In the CRF, the function $\omega$ is given by the nonnegative function $\Psi$.

    \item When we are concerned with numerical stability, or we only need the logarithm of the quantities of interest, we can use the \textit{log-real semiring}. The values $\alpha(v)$ and $\beta(v)$ now give the log-inside and log-outside values respectively. In particular, the value $\alpha$ at the root now gives the lognormalizer $\log Z(\x)$. In the CRF, the function $\omega$ is given by the logarithm of $\Psi$.

    \item When we intantiate the inside recursion with the \textit{max-tropical semiring} we get the recursion that computes the weight of the derrivation with the maximum weight. This tree is, of course, the Viterbi tree\footnote{Hence the name \textit{Viterbi semiring}}.

    \item The Viterbi semiring derives the \textit{weight} of the best tree. Replacing the max in the Viterbi semiring with an argmax derives the best tree itself. Roughly spearking, because although this idea can be made precise by a constructing the \textit{Viterbi-derivation semiring} \citep{goodman1999semiring} it is a litle more complicated than that\footnote{And most of all, a bit cumbersome.}.

\end{itemize}
