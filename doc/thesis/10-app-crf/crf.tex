% \bibliography{../src/bibliography}

In the derrivation of the inference algorithms I make use of the notion of a \textit{weighted hypergraph} as a compact representation of all parses and their scores \citep{gallo1993directed,klein2004parsing}. I also use some of the ideas and notation of semiring parsing \citep{goodman1999semiring,eisner2009semirings}.

I also introduce the following notation for termporary reasons because I cannot make up my mind about how to write these things. Let $\y_a = (i, j, \ell)$, then:
\begin{align*}
  A &= \ell  \\
  \psi(A, i, j) &= \Psi(\x, \y_a)
\end{align*}

\section{Hypergraph}
  A backward-hypergraph \citep{gallo1993directed}, or simply \textit{hypergraph} is a pair $G = (V, E)$, consisting of a set of nodes $V$, and a set of directed hyperedges $E \subseteq 2^V \times V$ that connect a \textit{set} of nodes at the \textit{tail} of the arrow to a single node at the \textit{head} of the arrow.

  In our case the hypegraph represents all the possible derivations of a sentence under a grammar. The $V$ consists of the invidual words of a sentence $x$, together with the all posible labeled spans over that sentence:
  \begin{align*}
    V = \Big\{ \x_1, \dots, \x_n \Big\} \cup \Big\{ (A, i, j) \mid A \in \Lambda, 0 \leq i < j \leq n \Big\} \cup \Big\{ (S^{\dagger}, 0, n) \Big\}.
  \end{align*}
  The edge set describes all the ways adjacent constituents can be combined to form a larger constituent. Because we (implicitly) use a normal form grammar containing all possible productions, the set of hyperedges is particularly regular: $E$ contains all edges that connect nodes $(B, i, k)$ and $(C, k, j)$ at the tail with $(A, i, j)$ at the head for all $0 \leq i < k < j \leq n$, and all edges that connect $w_i$ to $(A, i, i+1)$:

  \begin{align*}
    E
      &= \Bigg\{ \Big\langle \Big\{ (B, i, k), (C, k, j) \Big\},  (A, i, j) \Big\rangle \; \Bigg\vert \; A, B, C \in \Lambda, \; 0 \leq i < k < j \leq n \Bigg\}  \\
      &\quad\cup \Bigg\{ \Big\langle \{ w_i \}, (A, i, i+1) \Big\rangle \; \Bigg\vert \; A \in \Lambda, \; 0 \leq i < n \Bigg\}  \\
  \end{align*}

  Figure \ref{fig:hypergraph} shows a fragment of such a hypergraph for an example sentence. Finally, a weighted hypergraph is a hypergraph equided with a fuction $\omega : E \to \reals$ that assigns a weight to each edge; in our case this is the function $\Psi$.

\begin{figure}[h]
  \center
  \begin{tikzpicture}[scale=.6]
    % \documentclass[11pt]{article}
% \usepackage[upright]{fourier}
% \usepackage{tikz}
% \usepackage{amsmath,amssymb,amsfonts}
%
% \usetikzlibrary{arrows,calc}

% \newcommand{\boundellipse}[3]% center, x rad, y rad
% {(#1) ellipse [x radius=#2,y radius=#3]
% }

% \begin{document}

% \tikzstyle{every node}=[circle, draw, inner sep=0pt, minimum size=7mm]

% \tikzstyle{every node}=[circle, draw, inner sep=0pt, minimum size=11mm, node distance =1 cm and 1cm ]

% \begin{tikzpicture}

\node(a){\textit{The}} ;
\node(b) at ($(a)+(2.5,0)$){\textit{very}} ;
\node(c) at ($(b)+(2.5,0)$){\textit{hungry}} ;
\node(d) at ($(c)+(2.5,0)$){\textit{cat}} ;
\node(e) at ($(d)+(2.5,0)$){\textit{meows}} ;
\node(f) at ($(e)+(2.5,0)$){\textit{.}} ;

\node(A) at ($(a)+(0,2)$){$\varnothing_{0}^{1}$} ;
\node(B) at ($(A)+(2.5,0)$){$\varnothing_{1}^{2}$} ;
\node(C) at ($(B)+(2.5,0)$){$\varnothing_{2}^{3}$} ;
\node(D) at ($(C)+(2.5,0)$){$\varnothing_{3}^{4}$} ;
\node(E) at ($(D)+(2.5,0)$){$\text{VP}_{4}^{5}$} ;
\node(F) at ($(E)+(2.5,0)$){$\varnothing_{5}^{6}$} ;

\node(ADJP) at ($(B)+(0.8,2)$){$\text{ADJP}_{1}^{3}$} ;

\node(EMPTY) at ($(ADJP)+(0.5,2)$){$\varnothing_{1}^{4}$} ;
\node(NP) at ($(ADJP)+(2,2.5)$){$\text{NP}_{1}^{4}$} ;

\node(NP2) at ($(A)+(2,7)$){$\text{NP}_{0}^{4}$} ;

\node(EMPTY2) at ($(b)+(1.5,11)$){$\varnothing_{0}^{5}$} ;
\node(S) at ($(c)+(0,14)$){$\text{S}_{0}^{6}$} ;
\node(TOP) at ($(S)+(0,2)$){$\text{S}^{\dagger}_{0}^{6}$} ;


\draw[->] (a) to [in=-90,out=90](A);
\draw[->] (b) to [in=-90,out=90](B);
\draw[->] (c) to [in=-90,out=90](C);
\draw[->] (d) to [in=-90,out=90](D);
\draw[->] (e) to [in=-90,out=90](E);
\draw[->] (f) to [in=-90,out=90](F);

\draw[->] (B) to [in=-90,out=90](ADJP);
\draw[->] (C) to [in=-90,out=90](ADJP);

\draw[->] (ADJP) to [in=-90,out=90](EMPTY);
\draw[->] (D) to [in=-90,out=90](EMPTY);

\draw[dashed,->] (ADJP) to [in=-90,out=90](NP);
\draw[dashed,->] (D) to [in=-90,out=90](NP);

\draw[->] (A) to ++(-0.2,3) to [in=-90,out=90](NP2) ;
\draw[->] (EMPTY) to [in=-90,out=90](NP2) ;

\draw[dashed,->] (A) to [in=-90,out=90](NP2) ;
\draw[dashed,->] (NP) to [in=-90,out=90](NP2) ;

\draw[->] (NP2) to [in=-90,out=90](EMPTY2);
\draw[->] (E) to [in=-90,out=90](EMPTY2);

\draw[->] (EMPTY2) to [in=-90,out=90](S);
\draw[->] (F) to [in=-90,out=90](S);
\draw[->] (S) to [in=-90,out=90](TOP);

% \end{tikzpicture}
%
% \end{document}

  \end{tikzpicture}
  \caption{A fraction of a parse hypergraph showing two possible parses.}
  \label{fig:hypergraph}
\end{figure}


\section{Semiring}
  A semiring is an algebraic structure
  \begin{align*}
    \mathcal{K} = \langle \mathbb{K}, \oplus, \otimes, \bar{0}, \bar{1} \rangle,
  \end{align*}
  over a field $\mathbb{K}$, with additive and multiplicative operations $\oplus$ and $\otimes$, and additive and multiplicative identities $\bar{0}$ and $\bar{1}$. A semiring  is equivalent to a ring\footnote{The real numbers with regular addition and multiplication is an example of a ring.}, without the requirement of an additive inverse for each element. We use semirings to compute the quanitites over the weighted hypergraph by defining the function $\omega$ over its field $\mathbb{K}$ and by accumulating these weights with the binary operations. In particular we will accumulate quantities in logarithmic domain, and use the \textit{log semiring}
  \begin{align}
    \langle \reals \cup \{ -\infty \}, \text{logaddexp}, +, -\infty, 0 \rangle
  \end{align}
  to accumulate log-potentials, and the \textit{viterbi semiring}
  \begin{align}
    \langle \reals \cup \{ -\infty \}, \max, +, -\infty, 0 \rangle
  \end{align}
  to compute the weight of the highest scoring tree.


  % \section{Semiring formulation}
  % So yeah, the highlights are: an edge connects three nodes, a parent and two \textsc{children}, each node is a labelled \textsc{span}; you need to identify the scoring function for an edge, let’s call it $w(e)$,  in this case we have
  % \begin{align}
  %     w(e) = f(\textsc{head}(e)) \bigotimes_{c \in \textsc{children}(e)} g(\textsc{span}(c))
  % \end{align}
  % where $f$ and $g$ are parametric functions; then you can compute the Inside recursion for a node v
  % \begin{align}
  %     I(v) = \bigoplus_{e \in BS(v)} w(e) \otimes \bigotimes_{c \in \textsc{children}(e)} I(c)
  % \end{align}
  % where I’m using $BS(v)$ to denote the set of edges incoming to $v$; note that BS here basically enumerates the different ways to segment the string under $(i,j)$ into two adjacent parts and the different labels of each child \textsc{span} (let’s call these $a$ and $b$, each an element in the labelset $L$), thus we can write
  % \begin{align}
  % I(v=[i,j,l]) = \bigoplus_{ \substack{ e=[i,j,l,k,a,b]: \\ a \in L, \\ b \in L, \\ k \in \{i+1,...,j-1\} } } w(e) \otimes I([i,k,a]) \otimes I([k+1,j,b])
  % \end{align}
  % Now the key is to realise that $w(e)$ factorises and therefore we can rewrite this as
  % \begin{align}
  %     I(v=[i,j,l]) = f(i,j,l) &\otimes \bigoplus_{k=i+1}^{j-1} g(i,k) \otimes g(k+1,j) \\
  %         &\otimes \bigoplus_{a \in L}  I([i,k,a]) \\
  %         &\otimes \bigoplus_{b\in L} I([k+1,j,b])
  % \end{align}
  % and this finally motivates having an inside table for the \textsc{span}s (with labels summed out), let’s call that
  % \begin{align}
  %     S(i,j) = \bigoplus_{l \in L} I(i,j,l)
  % \end{align}
  % and then we have the result
  % \begin{align}
  % \label{eq:inside-semiring}
  %     I(i,j,l) = f(i,j,l)\otimes \bigoplus_{k=i+1}^{j-1} g(i,k)\otimes g(k+1,j) \otimes S(i,k) \otimes S(k+1,j).
  % \end{align}
