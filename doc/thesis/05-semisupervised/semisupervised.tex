% \bibliography{../src/bibliography.bib}

In this chapter we show how the RNNG can be trained on unlabeled data. Together with the regular, supervised, objective, this derives a way to perform semisupervised training.
\begin{itemize}
  \item I formulate an unsupervised objective for the RNNG that we can combine with the supervised objective to perfom semisupervised training.
  \item I introduce an approximate posterior in the form of a discriminative parser and derive a variational lower bound on the unsupervised objective.
  \item I show how to obtain gradients for this lowerbound by rewrinting the gradient into a form that is called the score function estimator \citep{williams1992reinforce,fu2006gradient}.
\end{itemize}

\paragraph{Notation} In this chapter we write $\x$ for a sentence, $\y$ for a (latent) constituency tree, and $\yieldx$ for the \textit{yield} of $\x$, all trees that can be assigned to $\x$. Furthermore, let $\mathcal{D}_L = \{ (\x_i, \y_i)\}_{i=1}^N$ be a labeled dataset of sentences $\x$ with gold trees $\y$, and let $\mathcal{D}_U = \{ \x_i \}_{i=1}^M$ be an unlabeled dataset consisting of just sentences $\x$. We denote our generative RNNG with $\ptheta$ and when we write $\qlambda$ we will mean either the discriminative RNNG or the CRF parser.

\section{Objective}
We define the following general semi-supervised objective
\begin{align*}
  \mathcal{L}(\theta, \lambda) \defeq \mathcal{L}_{S}(\theta) + \mathcal{L}_{U}(\theta, \lambda).
\end{align*}
The supervised objective $\mathcal{L}_{S}$ is optimized over $\mathcal{D}_L$ and $\mathcal{L}_{U}$ the unsupervised objective optimized over $\mathcal{D}_U$. We introduce $\alpha \in \mathbb{R}_{\geq 0}$ as an arbitrary scalar controlling the contribution of the unsupervised objective.

\paragraph{Supervised objective}
We define the supervised objective $\mathcal{L}_{S}(\theta)$ as
\begin{align*}
  \mathcal{L}_{S}(\theta)
    &\defeq \sum_{(\x, \y) \in \mathcal{D}_L} \log \ptheta (\x, \y) \\
\end{align*}
This objective is optimized as usual using stochastic gradient estimates:
\begin{align*}
  \nabla_{\theta} \mathcal{L}_{S}(\theta)
    &\approx \frac{N}{K} \sum_{(\x, \y) \in \mathcal{B}} \nabla_{\theta} \log \ptheta (\x, \y), \\
\end{align*}
where $\mathcal{B} \subseteq \mathcal{D}_U$ is a mini-batch of size $K$ sampled uniformly from the dataset. We rely on automatic differentiation to compute $\nabla_{\theta} \log \ptheta(\x, \y)$ \citep{baydin2017automatic}.

\paragraph{Unsupervised objective}
We define the unsupervised objective $\mathcal{L}_{U}(\theta, \lambda)$ as
\begin{align*}
  \mathcal{L}_{U}(\theta, \lambda)
    &\defeq \sum_{\x \in \mathcal{D}_U} \log p (\x) \\
    &= \sum_{\x \in \mathcal{D}_U} \log \sum_{ \y \in \mathcal{Y}(\x)} \ptheta(\x, \y)
\end{align*}
This is a language modelling objective, in which we treat $\y$ as latent. We have noted the a consequence the lack of independence assumptions of the RNNG is that the sum over trees $\y$ is not tractable. To optimize this objective we must thus fall back on approximate methods.

\section{Variational approximation} We optimize the unsupervised objective using variational inference \citep{blei2016vi}. We introduce a posterior $\qlambda(\y | \x)$ parametrised by $\lambda$ and use Jensen's inequality to derive a variational lower bound on the objective $\mathcal{L}_U(\theta, \lambda)$. First we bound the likelihood of one $\x$ in $\mathcal{D}_U$
\begin{align*}
  \log p (\x)
    &= \log \sum_{\y  \in \mathcal{Y}(\x)} \qlambda(\y |\x) \frac{\ptheta(\x,\y )}{\qlambda(\y | \x)} \\
    &= \log \expect_{q} \bigg[\frac{\ptheta(\x,\y )}{ \qlambda(\y | \x)} \bigg] \\
    &\geq \expect_{q} \bigg[\log \frac{\ptheta(\x,\y )}{\qlambda(\y | \x)} \bigg] \\
    &= \expect_{q} \Big[\log \ptheta(\x,\y )  - \log \qlambda(\y | \x) \Big]
  \label{eq:lowerbound}
\end{align*}
and write
\begin{align}
  \elbo(\theta, \lambda)
   &\defeq \sum_{\x \in \mathcal{D}_U} \expect_{q} \Big[\log \ptheta(\x,\y )  - \log \qlambda(\y | \x) \Big] \nonumber \\
   &\leq \sum_{\x \in \mathcal{D}_U} \log p (\x) \nonumber \\
   &= \mathcal{L}_{U}(\theta, \lambda)
\end{align}
as a lower bound on our true objective $\mathcal{L}_{U}$. This is a particular instance of the evidence lower bound (ELBO) \citep{blei2016vi}. The quantity can be rewritten to reveal an entropy term H($q$)
\begin{align*}
  \expect_{q} \Big[\log \ptheta(\x,\y )  - \log  \qlambda(\y | \x) \Big]
    &=\expect_{q} \Big[\log \ptheta(\x,\y ) \Big]  - \expect_{q} \Big[\log \qlambda(\y | \x) \Big] \\
    &= \expect_{q} \Big[\log \ptheta(\x,\y ) \Big]  + \text{H}(q),
  \label{eq:expectation}
\end{align*}
which gives this objective an intuitive interpretation. On the one hand, the objective aims to

\paragraph{Posterior}
The posterior $q$ can be any kind of models, with the only condition that for all $ \x $ and $ \y \in \yieldx$,
\begin{equation*}
  p(\x, \y) > 0 \Rightarrow q( \y| \x ) > 0.
\end{equation*}
This condition is fulfilled by any discriminatively trained parser with the same support as the joint RNNG $p$. We have two obvious choices at hand: the discriminatively trained RNNG, and the CRF parser that we introduced in chapter \ref{04-crf}.

An interesting advantage of the CRF parser is that we \textit{can} compute the entropy H$(q)$ exactly. This contrasts with the discriminative RNNG, where H$(q)$ can only be approximated. To make this explicit we introduct separate ELBO objectives:
\begin{align}
  \elbo_{\textsc{rnng}}(\theta, \lambda) &\defeq \sum_{\x \in \dataset_U}\expect_{q} \Big[\log \ptheta(\x,\y )  - \log  \qlambda(\y | \x) \Big] \\
  \elbo_{\textsc{crf}}(\theta, \lambda) &\defeq \sum_{\x \in \dataset_U} \expect_{q} \Big[\log \ptheta(\x,\y ) \Big]  + \text{H}(q).
\end{align}


\section{Optimization}
Just like the supervised objective $\mathcal{L}_U$ we optimize the lower bound $\elbo$ by gradient optimization, which means that we need to compute the gradients $\nabla_{\theta} \elbo(\theta, \lambda)$ and $\nabla_{\lambda} \elbo(\theta, \lambda)$.

\paragraph{Gradients of joint parameters} The first gradient is easy and permits a straightforward Monte-Carlo estimate:
\begin{align*}
  \nabla_{\theta} \elbo(\theta, \lambda)
    &= \nabla_{\theta} \expect_{q} \Big[\log \ptheta(\x, \y)  - \log \qlambda(\y | \x) \Big] \\
    &= \expect_{q} \Big[ \nabla_{\theta} \log \ptheta(\x, \y) \Big] \\
    &\approx \frac{1}{K}\sum_{i=1}^K  \nabla_{\theta} \log \ptheta(\x, \y_{i})
\end{align*}
where $y_i \sim \qlambda( \cdot | \x)$ for $i=1,\dots,K$ are samples from the approximate posterior. We can move the gradient inside the expectation because $q$ does not depend on $\theta$, and note that $\nabla_{\theta} \log \qlambda(\y | \x) = 0$.

\paragraph{Gradients of posterior parameters}
The second gradient is not so straightforward and requires us to rewrite the objective into a form that is called the \textit{score function estimator} \citep{fu2006gradient}. Firstly we define a \textit{learning signal}
\begin{equation}
  L(\x, \y) \defeq \log \ptheta(\x, \y) - \log \qlambda(\y | \x),
\end{equation}
and use the identity in equation \ref{eq:score-function-estimator} that we derive in the appendix
\begin{align*}
  \nabla_{\lambda} \elbo(\theta, \lambda)
    &= \nabla_{\lambda} \expect_{q} \Big[ L(\x, \y) \Big] \\
    &= \expect_{q} \Big[ L(\x, \y) \nabla_{\lambda} \log \qlambda(\y | \x) \Big].
\end{align*}
In this rewritten form the gradient is in the form of an expectation, and that does permit a straightforward MC estimate:
\begin{align}
    \expect_{q} \Big[ L(\x, \y) \nabla_{\lambda} \log \qlambda(\y | \x) \Big]
        &\approx \frac{1}{K}\sum_{i=1}^K  L(\x, \y_i)\nabla_{\lambda} \log \qlambda(\x | \y_i)
\end{align}
where again $y_i \sim \qlambda(\cdot | \x)$ for $i=1,\dots,K$ are independently sampled from the approximate posterior. This estimator has been derived in slightly different forms in \citet{williams1992reinforce,paisley2012viss,mnih2014nvil,ranganath2014black,miao2016discrete} and is also known as the \textsc{reinforce} estimator \citep{williams1992reinforce}.


\section{Variance reduction}
We introduce the two baselines:
\begin{itemize}
  \item Feedforward baseline \citep{miao2016discrete}.
  \item Argmax baseline from \citet{rennie2017argmax} which is exact in the CRF, and approximate in the RNNG.
  \item The CRF has no variance in estimating the entropy.
\end{itemize}


\section{Experiments}
\begin{itemize}
  \item Experiments with the two baselines and the two posteriors.
  \item Compare to a simple baseline: supervised learing on mixed gold-silver trees (partially predicted).
  \item Analyze the variance reduction provided by the different baselines.
\end{itemize}


\section{Related work}
\begin{itemize}
  \item Discrete latent variables in neural models \citep{miao2016discrete,yin2018structvae}.
  \item Semisupervised training for the RNNG \citep{cheng2017rnng}.
  \item Argmax baseline \cite{rennie2017argmax}.
\end{itemize}
